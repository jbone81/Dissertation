\begin{appendices}

  \addtocontents{toc}{\protect\renewcommand{\protect\cftchappresnum}{\appendixname\space}}
  \addtocontents{toc}{\protect\renewcommand{\protect\cftchapnumwidth}{7em}}

  \chapter{Survey Question to Research Question Mapping}\label{appendix:survey-mapping}

  This appendix provides traceability between survey questions and research questions, following the systems engineering lifecycle approach outlined in the dissertation methodology. The survey structure addresses three core dimensions---\textbf{Awareness}, \textbf{Applicability}, and \textbf{Perceived Value}---as they pertain to Digital Engineering and its four pillars: Model-Based Systems Engineering, the Digital Thread, Digital Twin, and Product Lifecycle Management.

  \section{Research Questions}

  \begin{table}[H]
    \centering
    \caption{Research Questions}\label{tab:research-questions}
    \begin{tabular}{@{}cp{5.5in}@{}}
      \toprule
      \textbf{RQ} & \textbf{Research Question} \\
      \midrule
      RQ1 & To what extent are IT and information assurance professionals aware of Digital Engineering capabilities, including MBSE, the Digital Thread, digital twin technologies, and PLM principles? \\
      \midrule
      RQ2 & Do IT and information assurance professionals perceive Digital Engineering capabilities as potentially valuable or important for their work in IA, security compliance, and IT service delivery? \\
      \midrule
      RQ3 & Do IT and information assurance professionals believe DE practices would help them perform their jobs, meet compliance requirements, or enhance organizational capabilities? \\
      \bottomrule
    \end{tabular}
  \end{table}

  \section{Survey Structure Aligned to Core Dimensions}

  \begin{table}[H]
    \centering
    \caption{Survey Section Structure}
    \label{tab:survey-structure}
    \begin{tabular}{@{}lllp{2.5in}@{}}
      \toprule
      \textbf{Section} & \textbf{Core Dimension} & \textbf{Primary RQ} & \textbf{Description} \\
      \midrule
      Section 1 & Awareness & RQ1 & Baseline familiarity and professional exposure \\
      \midrule
      Section 2 & Awareness & RQ1 & Understanding of specific DE capabilities/pillars \\
      \midrule
      Section 3 & Applicability & RQ2, RQ3 & Perceived relevance to IT and IA domains \\
      \midrule
      Section 4 & Perceived Value & RQ2, RQ3 & Value assessment for IT operations \\
      \midrule
      Section 5 & Perceived Value & RQ2, RQ3 & Value assessment for IA/Cybersecurity operations \\
      \midrule
      Section 6 & Demographics & Supporting & Professional field and experience level \\
      \bottomrule
    \end{tabular}
  \end{table}

  \section{Section 1: Awareness and Familiarity with Digital Engineering}

  \textbf{Core Dimension: Awareness} 
  
  \textbf{Primary Research Question --- RQ1}

  \begin{table}[H]
    \centering
    \caption{Section 1 Question Mapping}
    \label{tab:section1-mapping}
    \small
    \begin{tabular}{@{}cp{2.2in}p{1in}ll@{}}
      \toprule
      \textbf{Q\#} & \textbf{Question Text} & \textbf{Format} & \textbf{DE Pillar} & \textbf{RQ} \\
      \midrule
      1.1 & Please rate your level of familiarity with Digital Engineering concepts and practices. & 5-point Likert (Familiarity) & All/General & RQ1 \\
      \midrule
      1.2 & Have you encountered Digital Engineering methodologies, frameworks, or tools in your professional work within the past two years? & Binary (Yes/No) & All/General & RQ1 \\
      \bottomrule
    \end{tabular}
  \end{table}
\begin{figure}
    \centering
    \includegraphics[width=1\linewidth]{Figures/OverviewDiagramOfPackageSurveyQuestions.png}
    \caption{Taxonomy of Survey Questions}
    \label{fig:SurveyTaxonomy}
\end{figure}
\subsection{Rationale}
  
These questions establish baseline awareness metrics essential to all subsequent analysis. Question 1.1 measures self-assessed familiarity (theoretical awareness), while Question 1.2 measures practical professional exposure (applied awareness). Together they distinguish between those who have merely heard of Digital Engineering and those who have encountered it in the course of their professional duties.

\section{Section 2: Understanding of Digital Engineering Capabilities}

  \textbf{Core Dimension: Awareness}
  
  \textbf{Primary Research Question --- RQ1}

  \begin{table}[H]
    \centering
    \caption{Section 2 Question Mapping}
    \label{tab:section2-mapping}
    \small
    \begin{tabular}{@{}cp{2.8in}p{0.9in}p{0.9in}l@{}}
      \toprule
      \textbf{Q\#} & \textbf{Question Text} & \textbf{Format} & \textbf{DE Pillar} & \textbf{RQ} \\
      \midrule
      2.1 & DE includes model-based systems engineering approaches that can improve development processes. & 5-point Likert (Agreement) & MBSE & RQ1 \\
      \midrule
      2.2 & DE can enable digital twin development and virtual prototyping for IT systems. & 5-point Likert (Agreement) & Digital Twin & RQ1 \\
      \midrule
      2.3 & DE supports continuous integration and data-driven decision-making in technology development. & 5-point Likert (Agreement) & Digital Thread & RQ1 \\
      \midrule
      2.4 & DE enables digital twin technology that can simulate security scenarios and test defensive measures without impacting production systems. & 5-point Likert (Agreement) & Digital Twin (Security) & RQ1 \\
      \midrule
      2.5 & DE supports continuous security validation and data-driven threat analysis throughout the development lifecycle. & 5-point Likert (Agreement) & Digital Thread (Security) & RQ1 \\
      \midrule
      2.6 & DE can improve security control implementation through automated compliance checking and verification. & 5-point Likert (Agreement) & PLM / Traceability & RQ1 \\
      \bottomrule
    \end{tabular}
  \end{table}

\subsection{Rationale} 
  
This section probes understanding of specific Digital Engineering pillars applied to IT and security contexts. Questions address all four pillars: MBSE (Q2.1), Digital Twin (Q2.2, Q2.4), the Digital Thread (Q2.3, Q2.5), and PLM with its authoritative traceability capabilities (Q2.6). The inclusion of both general IT applications (Q2.1-Q2.3) and security-specific applications (Q2.4-Q2.6) enables meaningful comparison across professional domains.

\section{Section 3: Applicability of Digital Engineering}

\textbf{Core Dimension: Applicability}

\textbf{Primary Research Questions --- RQ2, RQ3}

\begin{table}[H]
    \centering
    \caption{Section 3 Question Mapping}
    \label{tab:section3-mapping}
    \small
    \begin{tabular}{@{}cp{2.2in}p{0.9in}p{0.8in}l@{}}
      \toprule
      \textbf{Q\#} & \textbf{Question Text} & \textbf{Format} & \textbf{DE Pillar} & \textbf{RQ} \\
      \midrule
      3.1 & DE methodologies have relevant applications within the information technology sector. & 5-point Likert (Agreement) & All/General & RQ2 \\
      \midrule
      3.2 & DE methodologies have relevant applications for addressing information assurance challenges. & 5-point Likert (Agreement) & All/General & RQ2 \\
      \midrule
      3.3 & The ability to utilize digital twins to test changes against accurate replicas of production environments would provide value to my organization. & 5-point Likert (Agreement) & Digital Twin & RQ2, RQ3 \\
      \midrule
      3.4 & The use of digital models to map and document an organization's environment and configurations would provide value to my organization. & 5-point Likert (Agreement) & MBSE & RQ2, RQ3 \\
      \midrule
      3.5 & The use of digital lifecycle management to meet compliance and service delivery requirements would provide value to my organization. & 5-point Likert (Agreement) & PLM & RQ2, RQ3 \\
      \midrule
      3.6 & My organization faces regulatory or compliance requirements that could benefit from Digital Engineering approaches. & 5-point Likert (Agreement) & All/General & RQ2, RQ3 \\
      \bottomrule
    \end{tabular}
  \end{table}

  \subsection{Rationale}
  
  This section bridges awareness and value by examining whether respondents perceive Digital Engineering as applicable to their professional contexts. Questions 3.1 and 3.2 assess domain-level applicability (IT versus IA), while Questions 3.3 through 3.5 assess pillar-specific organizational value (Digital Twin, MBSE, PLM). Question 3.6 identifies compliance-driven need, which stands central to the research focus upon Information Assurance and regulatory compliance.

  \section{Section 4: Value Assessment for Information Technology}

  \textbf{Core Dimension Perceived Value (IT Domain)}
  
  \textbf{Primary Research Questions --- RQ2, RQ3}

  \begin{table}[H]
    \centering
    \caption{Section 4 Question Mapping}
    \label{tab:section4-mapping}
    \small
    \begin{tabular}{@{}cp{2in}p{0.9in}p{1in}l@{}}
      \toprule
      \textbf{Q\#} & \textbf{Question Text} & \textbf{Format} & \textbf{Value Category} & \textbf{RQ} \\
      \midrule
      4.1 & DE could deliver meaningful value to my organization's information technology processes. & 5-point Likert (Agreement) & Overall IT Value & RQ2, RQ3 \\
      \midrule
      4.2 & DE could reduce development cycle time in my organization. & 5-point Likert (Agreement) & Efficiency Benefit & RQ3 \\
      \midrule
      4.3 & DE could improve product quality and reduce defects in my organization. & 5-point Likert (Agreement) & Quality Benefit & RQ3 \\
      \midrule
      4.4 & DE could improve collaboration effectiveness across development teams in my organization. & 5-point Likert (Agreement) & Collaboration Benefit & RQ3 \\
      \midrule
      4.5 & My organization would be willing to invest in DE capabilities if clear ROI could be demonstrated. & Ternary (Yes/No/ Unsure) & Investment Willingness & RQ2 \\
      \bottomrule
    \end{tabular}
  \end{table}

  \subsection{Rationale}
  
  This section measures IT-specific value perceptions. Question 4.1 provides an overall IT value assessment. Questions 4.2 through 4.4 examine specific operational benefits---efficiency, quality, and collaboration---that directly relate to job performance as addressed by RQ3. Question 4.5 measures organizational adoption interest, indicating whether perceived value rises to a level sufficient to warrant investment.

  \section{Section 5: Value Assessment for Information Assurance}

  \textbf{Core Dimension: Perceived Value (Information Assurance Domain)}
  
  \textbf{Primary Research Questions --- RQ2, RQ3}

  \begin{table}[H]
    \centering
    \caption{Section 5 Question Mapping}
    \label{tab:section5-mapping}
    \small
    \begin{tabular}{@{}cp{2.5in}p{1.0in}p{1.0in}l@{}}
      \toprule
      \textbf{Q\#} & \textbf{Question Text} & \textbf{Format} & \textbf{Value Category} & \textbf{RQ} \\
      \midrule
      5.1 & DE could deliver meaningful value to my organization's information assurance and security operations. & 5-point Likert (Agreement) & Overall Security Value & RQ2, RQ3 \\
      \midrule
      5.2 & DE could reduce the time required to identify and remediate security vulnerabilities in my organization. & 5-point Likert (Agreement) & Vulnerability Mgmt Benefit & RQ3 \\
      \midrule
      5.3 & DE could improve security posture and reduce successful cyber incidents in my organization. & 5-point Likert (Agreement) & Security Posture Benefit & RQ3 \\
      \midrule
      5.4 & DE could enhance threat modeling and risk assessment capabilities in my organization. & 5-point Likert (Agreement) & Threat Modeling Benefit & RQ3 \\
      \midrule
      5.5 & DE could improve collaboration between security teams, development teams, and operations teams in my organization. & 5-point Likert (Agreement) & Cross-Team Collaboration & RQ3 \\
      \midrule
      5.6 & DE could help my organization achieve better compliance with security frameworks and regulatory requirements. & 5-point Likert (Agreement) & Compliance Benefit & RQ3 \\
      \midrule
      5.7 & My organization would be willing to invest in DE capabilities for information assurance purposes if clear ROI could be demonstrated. & Binary (Yes/No) & Investment Willingness & RQ2 \\
      \bottomrule
    \end{tabular}
  \end{table}

  \subsection{Rationale}
  
  This section measures information assurance-specific value perceptions. Question 5.1 provides an overall security value assessment. Questions 5.2 through 5.6 examine specific security operational benefits directly related to job performance and compliance requirements as addressed by RQ3. The emphasis upon compliance (Q5.6) directly addresses the dissertation's focus upon Information Assurance compliance frameworks. Question 5.7 measures security-specific investment willingness.

  \section{Section 6: Interest and Demographic Information}

  \textbf{Core Dimension: Supporting/Demographics}
  
  \textbf{Purpose}
  
  Enable subgroup analysis and assess future research/adoption interest

  \begin{table}[H]
    \centering
    \caption{Section 6 Question Mapping}
    \label{tab:section6-mapping}
    \small
    \begin{tabular}{@{}cp{2in}p{1.1in}p{1in}l@{}}
      \toprule
      \textbf{Q\#} & \textbf{Question Text} & \textbf{Format} & \textbf{Category} & \textbf{RQ} \\
      \midrule
      6.1 & Would you be interested in learning more about DE applications for information assurance and security operations in your industry? & Binary (Yes/No) & Learning Interest & Supporting \\
      \midrule
      6.2 & Would you recommend that your organization explore DE adoption for improving security operations? & Binary (Yes/No) & Recommendation & Supporting \\
      \midrule
      6.3 & Please identify your field of practice. & Categorical (IT/Security/ Engineering/Other) & Professional Field & Demographics \\
      \midrule
      6.4 & Please indicate your level of experience in your field of practice. & Categorical (Experience ranges) & Experience Level & Demographics \\
      \bottomrule
    \end{tabular}
  \end{table}

  \subsection{Rationale}
  
  Questions 6.1 and 6.2 measure forward-looking interest that complements value perception, indicating whether positive perceptions translate into desire for learning and willingness to recommend organizational exploration. Questions 6.3 and 6.4 enable comparative analysis between IT and security professionals and across experience levels, addressing whether awareness and perceptions vary systematically by professional background.

  \section{Summary: Question Distribution by Research Question}

  \begin{table}[H]
    \centering
    \caption{Question Distribution by Research Question}
    \label{tab:question-distribution-rq}
    \begin{tabular}{@{}lp{2in}p{1.2in}c@{}}
      \toprule
      \textbf{Research Question} & \textbf{Primary Questions} & \textbf{Supporting Questions} & \textbf{Total} \\
      \midrule
      RQ1 (Awareness) & Q1.1, Q1.2, Q2.1-Q2.6 & --- & 8 \\
      \midrule
      RQ2 (Perceived Value) & Q3.1, Q3.2, Q3.6, Q4.1, Q4.5, Q5.1, Q5.7 & Q6.1, Q6.2 & 10 \\
      \midrule
      RQ3 (Job/Compliance Help) & Q3.3-Q3.6, Q4.1-Q4.4, Q5.1-Q5.6 & --- & 14 \\
      \midrule
      Demographics & Q6.3, Q6.4 & --- & 2 \\
      \bottomrule
    \end{tabular}
  \end{table}

  Note: Several questions map to multiple research questions as they address both perceived value (RQ2) and anticipated benefits for job performance and compliance (RQ3).

  \section{Summary: Question Distribution by DE Pillar}

  \begin{table}[H]
    \centering
    \caption{Question Distribution by Digital Engineering Pillar}
    \label{tab:question-distribution-pillar}
    \begin{tabular}{@{}lp{4in}@{}}
      \toprule
      \textbf{DE Pillar} & \textbf{Questions} \\
      \midrule
      MBSE & Q2.1, Q3.4 \\
      \midrule
      Digital Twin & Q2.2, Q2.4, Q3.3 \\
      \midrule
      Digital Thread & Q2.3, Q2.5 \\
      \midrule
      PLM & Q2.6, Q3.5 \\
      \midrule
      General/All Pillars & Q1.1, Q1.2, Q3.1, Q3.2, Q3.6, Q4.1-Q4.5, Q5.1-Q5.7, Q6.1-Q6.4 \\
      \bottomrule
    \end{tabular}
  \end{table}

  \section{Analysis Framework}

  \subsection{Primary Analysis for Each Research Question}

  \subsubsection{RQ1 Analysis (Awareness):}
  \begin{itemize}
    \item Mean familiarity score (Q1.1) with 95\% confidence interval
    \item Percentage with professional exposure (Q1.2) with 95\% confidence interval
    \item Mean agreement scores for capability understanding (Q2.1-Q2.6)
    \item Percentage indicating agreement (\(\geq\)4) with each capability statement
    \item Comparison of awareness levels across professional fields (Q6.3) and experience levels (Q6.4)
  \end{itemize}

  \subsubsection{RQ2 Analysis (Perceived Value):}
  \begin{itemize}
    \item Mean agreement scores for applicability (Q3.1-Q3.2, Q3.6)
    \item Mean agreement scores for overall value (Q4.1, Q5.1)
    \item Percentage indicating investment willingness (Q4.5, Q5.7)
    \item Percentage indicating learning interest (Q6.1) and recommendation (Q6.2)
    \item Comparison of value perceptions across professional fields and experience levels
  \end{itemize}

  \subsubsection{RQ3 Analysis (Job/Compliance Help):}
  \begin{itemize}
    \item Mean agreement scores for specific benefits (Q4.2-Q4.4, Q5.2-Q5.6)
    \item Percentage agreeing with compliance benefits (Q3.5, Q3.6, Q5.6)
    \item Percentage agreeing with organizational capability benefits (Q3.3-Q3.5)
    \item Comparison of benefit perceptions across professional fields and experience levels
  \end{itemize}

  \subsection{Composite Scores}

  \begin{table}[H]
    \centering
    \caption{Composite Score Definitions}
    \label{tab:composite-scores}
    \begin{tabular}{@{}llcp{2in}@{}}
      \toprule
      \textbf{Composite} & \textbf{Questions} & \textbf{Items} & \textbf{Interpretation} \\
      \midrule
      Awareness Composite & Q1.1, Q2.1-Q2.6 & 7 & Higher = greater awareness/understanding \\
      \midrule
      IT Value Composite & Q3.1, Q3.3-Q3.5, Q4.1-Q4.4 & 8 & Higher = greater perceived IT value \\
      \midrule
      Security Value Composite & Q3.2, Q3.6, Q5.1-Q5.6 & 8 & Higher = greater perceived security value \\
      \bottomrule
    \end{tabular}
  \end{table}

  Internal consistency shall be assessed via Cronbach's alpha (acceptable if \(\alpha \geq 0.70\)).

  \section{Scale Reference}

  \subsection{Familiarity Scale (Q1.1)}

  \begin{table}[H]
    \centering
    \caption{Familiarity Scale}
    \label{tab:familiarity-scale}
    \begin{tabular}{@{}cll@{}}
      \toprule
      \textbf{Score} & \textbf{Label} & \textbf{Interpretation} \\
      \midrule
      1 & Not at all familiar & No awareness \\
      \midrule
      2 & Slightly familiar & Minimal awareness \\
      \midrule
      3 & Moderately familiar & Basic awareness \\
      \midrule
      4 & Very familiar & Good awareness \\
      \midrule
      5 & Extremely familiar & Expert awareness \\
      \bottomrule
    \end{tabular}
  \end{table}

  \subsection{Agreement Scale (Q2.1-Q5.6)}

  \begin{table}[H]
    \centering
    \caption{Agreement Scale}
    \label{tab:agreement-scale}
    \begin{tabular}{@{}cll@{}}
      \toprule
      \textbf{Score} & \textbf{Label} & \textbf{Interpretation} \\
      \midrule
      1 & Strongly disagree & Strong negative perception \\
      \midrule
      2 & Disagree & Negative perception \\
      \midrule
      3 & Neither agree nor disagree & Neutral/uncertain \\
      \midrule
      4 & Agree & Positive perception \\
      \midrule
      5 & Strongly agree & Strong positive perception \\
      \bottomrule
    \end{tabular}
  \end{table}

  \subsection{Experience Level Categories (Q6.4)}

  \begin{table}[H]
    \centering
    \caption{Experience Level Categories}
    \label{tab:experience-levels}
    \begin{tabular}{@{}lll@{}}
      \toprule
      \textbf{Category} & \textbf{Label} & \textbf{Career Stage} \\
      \midrule
      1-5 Years & Entry Level & Early Career \\
      \midrule
      6-10 Years & Mid-Level (Early) & Mid Career \\
      \midrule
      11-15 Years & Mid-Level (Late) & Mid Career \\
      \midrule
      16+ Years & Senior Level & Late Career \\
      \bottomrule
    \end{tabular}
  \end{table}

\end{appendices}