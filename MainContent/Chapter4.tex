\chapter[\leavevmode\newline Artifact Design and Analysis]{Artifact Design and Analysis}
\chaptermark{Artifact Design and Analysis}
\label{chap:Chapter_4}

This chapter presents the detailed design specification and analysis methodology for implementing Digital Engineering practices within information system security and assurance management. The proposed artifact demonstrates how Digital Engineering principles can enhance security visibility, security control implementation, and protection effectiveness.

\section{Artifact Overview and Design Philosophy}

The proposed artifact implements a comprehensive Digital Engineering framework that addresses the identified challenges in information system security management. This implementation integrates established security frameworks with Digital Engineering methodologies to create a cohesive approach to security management and control validation.

\section{Concept of Operations/Overview of The Artifact}

The development of an artifact that will provide evidence towards the research question requires a complex number of sub ``artifacts''. In support of the design and delivery of such a comprehensive security artifact, the decision to leverage standards-based tools from across a number of disciplines. The resultant aggregation of artifacts will result in a System Security Plan (SSP) and a Body of Evidence (BoE). 

\subsection{Standards Utilized}

