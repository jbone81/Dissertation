% \chapter*{Abstract}
\chapter*{\Large \uppercase{Abstract}}
\chaptermark{Abstract}
\label{chap:Abstract}
\addcontentsline{toc}{chapter}{Abstract}

%\begin{singlespace}

Digital Engineering has transformed how the Department of War, NASA, and the aerospace industry design, develop, and sustain complex systems. Its four pillars---Model-Based Systems Engineering, digital threads, digital twin technology, and Product Lifecycle Management---have delivered measurable improvements in mission assurance, configuration management, and lifecycle governance. The Unified Architecture Framework, now codified as ISO/IEC 19540, has emerged as the consolidating standard adopted by major defense organizations and commercial enterprises worldwide. Despite this proven operational value, these methods remain virtually untested within enterprise information technology and information assurance domains, where expanding compliance obligations---including the NIST Risk Management Framework and the DoW Cybersecurity Maturity Model Certification---impose documentation, traceability, and verification demands that current practices fail to sustain. Systematic literature review reveals that systems engineering, information assurance, cybersecurity, and IT service management evolved along independent academic trajectories, creating disciplinary silos that explain the near-complete absence of cross-disciplinary research. This research investigates whether IT and information assurance professionals recognize the potential that Digital Engineering capabilities hold for their work.

Grounded in a theoretical framework synthesizing Rogers' Diffusion of Innovations theory, the Technology Acceptance Model, and the Unified Theory of Acceptance and Use of Technology, the research targets IT and information assurance professionals across multiple sectors, enabling assessment of whether Digital Engineering awareness and perceived value vary by organizational context. A quantitative survey collects data across multiple dimensions: awareness, comprehension of specific capabilities, perceived applicability, and value assessments. This study establishes baseline empirical data on professional awareness and perceived value, furnishing an evidence foundation for strategic decisions regarding future research investment, adoption initiatives, and curricula development. Results shall inform both scholarly inquiry and practical advancement of mission assurance capabilities.


%\end{singlespace}