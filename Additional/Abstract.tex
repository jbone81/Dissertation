% \chapter*{Abstract}
\chapter*{\Large \uppercase{Abstract}}
\chaptermark{Abstract}
\label{chap:Abstract}
\addcontentsline{toc}{chapter}{Abstract}

%\begin{singlespace}


Digital Engineering has transformed how the Department of Defense, NASA, and the aerospace industry design, develop, and sustain complex systems. Its four pillars---Model-Based Systems Engineering, digital threads, digital twin, and Product Lifecycle Management---have delivered measurable improvements in mission assurance, configuration management, and lifecycle governance. The Unified Architecture Framework, now codified as ISO/IEC 19540, has emerged as the consolidating standard adopted by major defense organizations and commercial enterprises worldwide. Despite this proven operational value, these methods remain virtually untested within enterprise information technology and information assurance domains. This research investigates whether IT and information assurance professionals recognize the potential that Digital Engineering capabilities hold for their work.

This research targets IT and information assurance professionals across multiple sectors, enabling assessment of whether Digital Engineering awareness and perceived value vary by organizational context. The benefits demonstrated in defense and aerospace suggest logical application to organizations outside these sectors.

The research employs a quantitative survey methodology to collect data across multiple dimensions: awareness, comprehension of specific capabilities, perceived applicability, and value assessments. Systematic literature review documents a near-complete absence of academic research applying Digital Engineering methods to enterprise IT infrastructure or Information Assurance programs. This study establishes baseline empirical data regarding professional awareness and perceived value, furnishing an evidence foundation for strategic decisions regarding future research investment, industry adoption initiatives, and academic curricula development. These results shall inform both scholarly inquiry and practical advancement of mission assurance capabilities.

%\end{singlespace}